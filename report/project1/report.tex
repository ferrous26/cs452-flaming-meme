\documentclass[pdftex,10pt,a4paper]{article}
\usepackage{../cs452}

\begin{document}

\projectmake{1}

\section*{Overview}

This kernel development milestone adds the final set of kernel
features required for train control, namely serial I/O for both
the terminal and train controller. To demonstrate the finished kernel,
the M{\"a}rklin train set can be controlled by user input
to a terminal, and train track state is updated live on screen as the
trains travel around the track.

Commands issued into the terminal can manipulate trains and
switches, as well as monitor sensors. Trains can start, stop, toggle
lights and sound effects, and adjust speed. Switches can be toggled or
explicitly set to the straight or curved state.

The terminal displays information about current state
of the trains and switches, including a short history of the most
recent sensor activations. However, sensor activity cannot be
controlled by the user in any way.


\section*{Operation Instructions}

A pre-compiled kernel exists at
\ttt{/u/cs452/tftp/ARM/marada/k4.elf}, which can be loaded with
RedBoot using the following command:

\begin{center}
  \ttt{load -h 10.15.167.4 ARM/marada/k4.elf; go}
\end{center}

Notice that no offset should be specified for the load instruction.

The source code for the kernel exists in \ttt{/u3/marada/kernel4/},
and can be compiled with the following command chain:

\begin{center}
  \ttt{cd /u3/marada/kernel4 \&\& /u/cs444/bin/rake local}
\end{center}

Which will produce a \ttt{kernel.elf} file in the same directory as
the makefile, built in release mode with benchmarking enabled.

After Initialization the kernel will print
``\ttt{Welcome to Task Launcher}'' into the message logging area of
the screen and a command prompt will appear near the center of the
screen titled ``TERM> ''. At this point commands can be entered; keys
pressed should be echoed back at the command prompt in place of the
white cursor. Pressing the return key will confirm a command and cause
it to be processed. An online listing of the commands can be printed
by entering an empty command.

\subsection*{Submitted Files}
\begin{center}
\begin{tabular}{l|l}
  \bfseries File & \bfseries MD5 Hash
  \\\hline
  \csvreader[head to column names]{md5_info_headers.csv}{}%
  {\\\file & \ttt{\hash}}%
\end{tabular}
\end{center}
\newpage
\begin{center}
\begin{tabular}{l|l}
  \bfseries File & \bfseries MD5 Hash
  \\\hline
  \csvreader[head to column names]{md5_info_impls.csv}{}%
  {\\\file & \ttt{\hash}}%
\end{tabular}
\end{center}

\newpage
\section{FUUUU}

track data. we use what we were given, but with some modifications

track b has been remeasured

we use distance measurements for paramaters, because speed and
consequently time, will vary from train to train, and track piece to
track piece.

write about Putin

tweakables

calibration methods

pathing, including data format

only one train can be active at a time because of how sensors work

event based architecture

couriers, bitch. allow us to send things to other tasks without
blocking


\end{document}

%%% Local Variables:
%%% mode: latex
%%% TeX-master: t
%%% End:
