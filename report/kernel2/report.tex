\documentclass[pdftex,10pt,a4paper]{article}
\usepackage{../cs452}
\usepackage{verbatim}

\begin{document}

\kernelmake{2}

\section*{Overview}

Herp Derp


\section*{Kernel Design}

Kernel do thing and stuff. put the new stuff here!

\subsection*{Name Server}

The Nameserver is the first task that is started up on the initialization of
task launcher. Since the Nameserver is pivotal to the proper operation of the
OS if something goes wrong with the initialization of the Nameserver the task
launcher will abort causing the kernel to terminate. Due to the importance and
the requirement of all tasks to be able to communicate with it the Task Launcher
stores the Nameserver’s task id in a global variable that all tasks are able reach.

The Nameserver takes in a null terminated character sequence of no more than 8
characters (including the null terminator) and provides a querying task with
the task id that corresponds to that string. If a new task tries to register
the same name as another task that registered with the name server then the
lookup will now point to the task that registered most recently. The internal
storage of the server allows for up to 32 different lookups to be registered with it.

The API the Nameserver is split into 2 separate calls:
\begin{description}
\item[RegisterAs():] \hfill \\
This allows for a server to register itself with the name server for other tasks
find. Since this function is a wrapper around \ttt{Send()} it will return all of
the same errors as \ttt{Send()} and in addition will use -69 to express that the
server has no more space to store task mappings.

\item[WhoIs():] \hfill \\
Allows for a querying task to find the task id of a task that had registered
itself before hand. Similarly to \ttt{RegisterAs()}, this is a wrapper around
the \ttt{Send()} system call and has the same error codes but also contains
the error -69 which means that the lookup doesn’t lead to a valid task id.
\end{description}

The Nameserver stores all of the names into a single unsorted array and all of 
the tasks that each name into a separate array such the the index of both arrays
correspond to the same element. Since the name size is 8 characters or less the
server does word comparison though the name array until a match is found or all
of the currently registered names have been exhausted. The reason we used 8 as
the size is it allows for a sufficient large space of names and fits nicely into
two words so only two full comparisons need to be performed for every name
speeding up the liner search by a little while still maintaining a simple approach
for storage. Liner search was chosen due to its simplicity and the Nameserver
while important should only require use during the initialization or when having
a longer operating time is of little consequence.


\begin{center}
\begin{tabular}{l|l}
  \bfseries File & \bfseries MD5 Hash
  \\\hline
  \csvreader[head to column names]{md5_info.csv}{}%
  {\\\file & \ttt{\hash}}%
\end{tabular}
\end{center}


\section*{Rock Paper Scissors Output}

\verbatiminput{k2_task_out.txt}
(Note some blank lines have been removed)

\begin{description}
\item[Lines 1,2:] \hfill \\
As part of the Task Launchers initalization it calls \ttt{Create()} to start
up the RPS server. Since the RPS server has a higher priority than the task
launcher it then starts execution until it hits its \ttt{Receive()} call. At this
point the task launcher is the only ready task in the system and it continues on
until it busy waits for user input

\item[Lines 7-12:] \hfill \\
First Game

\item[Lines 17-22:] \hfill \\
Second Game

\item[Lines 27-32:] \hfill \\
Third Game

\item[Lines 37-42:] \hfill \\
Fourth Game
\end{description}


\end{document}

%%% Local Variables:
%%% mode: latex
%%% TeX-master: t
%%% End:
