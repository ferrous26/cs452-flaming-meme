\documentclass[pdftex,10pt,a4paper]{article}
\usepackage{../cs452}

\begin{document}

\kernelmake{3}

\section*{Overview}

This kernel development milestone adds handling for hardware
interrupts, a clock server, and system calls for event handling and
communication with the clock server. This milestone also includes a
task that demonstrates the clock servers abilities.

\section*{Operation Instructions}

A pre-compiled kernel exists at
\ttt{/u/cs452/tftp/ARM/marada/k3.elf}, which can be loaded with
RedBoot using the following command:

\begin{center}
  \ttt{load -h 10.15.167.4 ARM/marada/k3.elf; go}
\end{center}

Notice that no offset should be specified for the load instruction.

The source code for the kernel exists in \ttt{/u3/marada/kernel2/},
and can be compiled with the following command chain:

\begin{center}
  \ttt{cd /u3/marada/kernel2 \&\& source /u3/marada/.cs452 \&\& make clean \&\& make}
\end{center}

Which will produce a \ttt{kernel.elf} file in the same directory as
the makefile built in debug mode with minimal optimizations enabled.

After Initialization the kernel will respond with the message ``\ttt{Welcome to
Task Launcher}''.
At this point the key \ttt{1} will start the clock server
demonstration. The \ttt{h} key will print a list of other commands
which are available to demonstrate features added in this milestone.

\subsection*{Submitted Files}
\begin{center}
\begin{tabular}{l|l}
  \bfseries File & \bfseries MD5 Hash
  \\\hline
  \csvreader[head to column names]{md5_info.csv}{}%
  {\\\file & \ttt{\hash}}%
\end{tabular}
\end{center}

\newpage


\end{document}

%%% Local Variables:
%%% mode: latex
%%% TeX-master: t
%%% End:
