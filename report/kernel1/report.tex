\documentclass[pdftex,10pt,a4paper]{article}
\usepackage{../cs452}

\begin{document}

\kernelmake{1}

\section*{Overview}

Write something about the kernel (tasks) and using the Task Launcher here.

\section*{Operation Instructions}

A pre-compiled kernel exists at
\ttt{/u/cs452/tftp/ARM/marada/k1.elf}, which can be loaded with
default RedBoot command given in the course tutorial document:

\begin{center}
  \ttt{load -b 0x00218000 -h 10.15.167.4 ARM/marada/k1.elf; go}
\end{center}

The source code for my program exists in \ttt{/u3/marada/kernel1/},
which has the following list of files:


\section*{Kernel1 Task Output}

Kernels can do coolish things


\begin{center}
\begin{tabular}{l|l}
  \bfseries File & \bfseries MD5 Hash
  \\\hline
  \csvreader[head to column names]{md5_info.csv}{}%
  {\\\file & \ttt{\hash}}%
\end{tabular}
\end{center}

The program can be compiled with the following command:

\begin{center}
  \ttt{cd /u3/marada/kernel1 \&\& source /u3/marada/.cs452 \&\& make clean \&\& make}
\end{center}

Which will produce a \ttt{kernel.elf} file in the same directory as
the make file.

\section*{Structure}

The kernel is organized into three pieces: scheduler, tasks, and
support.

\subsection*{Task Descriptors}

Tasks metadata is stored in a static array of task descriptors. We
allow for up to 64 concurrent tasks. Each task descriptor contains the
task identifer \ttt{tid}, the parent task's task identifier
\ttt{p\_tid}, a priority level \ttt{priority}, a task index pointing
to the \ttt{next} scheduled task, some reserved space \ttt{reserved},
and the stack pointer for the current task \ttt{sp}.

The \ttt{tid} and \ttt{p\_tid} are stored as \ttt{int} types, while \ttt{priority}
and \ttt{next} are stored as \ttt{unsigned char} types, which
necessitates the \ttt{reserved} space of size \ttt{short
  int} in order to keep memory explicitly aligned. Finally, \ttt{sp}
is declared as a \ttt{unsigned int*}.

Task identifiers are assigned based on the array index of the task
descriptor. Each time that a descriptor is reused, the \ttt{tid} for
the descriptor is incremented by 64. This allows us to calculate the
array index for a descriptor in a single instruction given a
\ttt{tid}. It also trivializes the need to make sure two tasks never
have the same identifier. Ensuring that task identifiers never repeat
is not possible without infinite memory and computation time, however,
our strategy for allocating identifiers does maximize the available
number of identifiers given the constraints of the API. Furthermore,
task descriptors are allocated in a round robin fashion, using a
circular buffer for a free list, so that the entire 31 bit space is used.

The \ttt{p\_tid} identier is set at task allocation time as it is the
only time that the parent is guaranteed to be alive. The parent of a
task is always the task that was active when we entered the kernel to
handle the system call.

Tasks have 32 priority levels. Level 0 is the lowest priority, level
31 is the highest priority. The range of priority levels fits nicely
into a bit field that is one word in length. This allows us to
optimize the scheduler selection process by checking for the highest
set bit in the bit field, which can be done reasonably efficiently by
calculating the logarithm with base 2 of the bit field (and
subtracting 1). The range of levels is also more than enough for what
professor suggested in class.

The \ttt{next} index of the descriptor is used as part of the
scheduler for maintaining the priority queue list. Storing the
priority queue within the descriptor table saves memory space
and will hopefully let the cache perform better when it is evetually
enabled.

The \ttt{sp} is a pointer to the user stack where the task trap frame
is currently being stored. The \ttt{sp} is the minimum runtime
information that the descriptor needs to store. All other information
is stored on the user stack in the trap frame.

\subsection*{Scheduling}

\end{document}

%%% Local Variables:
%%% mode: latex
%%% TeX-master: t
%%% End:
