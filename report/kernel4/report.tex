\documentclass[pdftex,10pt,a4paper]{article}
\usepackage{../cs452}
\usepackage{verbatim}

\begin{document}

\kernelmake{4}

\section*{Overview}

This kernel development milestone adds the final set of kernel
features required for train control, namely serial I/O for both
the terminal and train controller. To demonstrate the finished kernel,
the M{\"a}rklin train set can be controlled by user input
to a terminal, and train track state is updated live on screen as the
trains travel around the track.

Commands issued into the terminal can manipulate trains and
switches, as well as monitor sensors. Trains can start, stop, toggle
lights and sound effects, and adjust speed. Switches can be toggled or
explicitly set to the straight or curved state.

The terminal displays information about current state
of the trains and switches, including a short history of the most
recent sensor activations. However, sensor activity cannot be
controlled by the user in any way.


\section*{Operation Instructions}

A pre-compiled kernel exists at
\ttt{/u/cs452/tftp/ARM/marada/k4.elf}, which can be loaded with
RedBoot using the following command:

\begin{center}
  \ttt{load -h 10.15.167.4 ARM/marada/k4.elf; go}
\end{center}

Notice that no offset should be specified for the load instruction.

The source code for the kernel exists in \ttt{/u3/marada/kernel4/},
and can be compiled with the following command chain:

\begin{center}
  \ttt{cd /u3/marada/kernel4 \&\& /u/cs444/bin/rake local}
\end{center}

Which will produce a \ttt{kernel.elf} file in the same directory as
the makefile, built in release mode with benchmarking enabled.

After Initialization the kernel will print
``\ttt{Welcome to Task Launcher}'' into the message logging area of
the screen and a command prompt will appear near the bottom of the
screen titled ``TERM> ''. At this point commands can be entered; keys
pressed should be echoed back at the command prompt. Pressing the
return key will confirm a command and cause it to be processed. An
online listing of the commands can be printed by using the ``help''
command.

\subsection*{Submitted Files}
\begin{center}
\begin{tabular}{l|l}
  \bfseries File & \bfseries MD5 Hash
  \\\hline
  \csvreader[head to column names]{md5_info.csv}{}%
  {\\\file & \ttt{\hash}}%
\end{tabular}
\end{center}

\newpage
\part*{The Kernel}

\section*{Task Descriptors}

\section*{Scheduling}

\section*{System Calls}

Just talk about the mechanism here, not any specific calls.

\subsection*{Core System Calls}

\subsubsection*{Create}
\subsubsection*{Pass}
\subsubsection*{Exit}
\subsubsection*{Abort}
\subsubsection*{Shutdown}

\subsection*{Message Passing}

\subsubsection*{Send}
\subsubsection*{Receive}
\subsubsection*{Reply}

\subsection*{Hardware Events}

\subsubsection*{AwaitEvent}

Event type details are given later. Here we just give the general API
details.

\subsection*{Message Passing Wrappers}

\subsubsection*{WhoIs}

And all the rest\ldots

\section*{Hardware Interrupts}

Just talk about the mechanism and practices. Mention how we handle all
hardware interaction in the kernel (except for Timer4).

\subsection*{UART1 Events}
\subsection*{UART2 Events}
\subsection*{Clock Events}

\section*{Tasks}

List all the different tasks we have, their purpose, and the priority
that they run at.

In the case of server tasks, we need to refer back to the wrapper
functions on how to interact.

\section*{Terminal Commands}

A full list of the commands that the terminal supports.

\section*{Abortions And Assertions}

This is where we talk about how we choose to detect and handle errors.

\section*{Known Bugs}

Clock UI stops updating at 100 minutes. Internal time continues to
increase.

\end{document}

%%% Local Variables:
%%% mode: latex
%%% TeX-master: t
%%% End:
